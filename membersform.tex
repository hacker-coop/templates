\documentclass[a4paper,14pt]{scrartcl}
\usepackage[utf8]{inputenc} 
\usepackage[pdftex,colorlinks=false]{hyperref}

\renewcommand{\MakeTextField}[2]{{\vbox to #2{\vfill\hbox to #1{\hrulefill}}}}

\newdimen\longline
\longline=\textwidth\advance\longline-4cm

\def\LayoutTextField#1#2{#2} % override default in hyperref

\def\lbl#1{\hbox to 4cm{#1\dotfill\strut}}%
\def\labelline#1#2{\lbl{#1}\vbox{\hbox{\TextField[bordercolor=1 1 1,name=#1,width=#2]{\null}}\kern2pt\hrule}}

\def\q#1{\hbox to \hsize{\labelline{#1}{\longline}}\vskip1.4ex}

\begin{document}

\section*{Personendaten Mitglieder}

% Mit diesem Formular sollen die Personendaten für Mitglieder erfasst
% werden wie sie auf dem geführten amtlichen Ausweis stehen.
% Ein Pseudony ist optional und kann vom Ausweis abweichen.

  \begin{Form}
    \q{Meldedatum}
    \q{Nachname}
    \q{Vorname}
    \q{Geburtsdatum}
    \q{Geburtsort}
    \q{Anschrift}
    \q{E-Mail}
    \q{Pseudonym}
   \end{Form}
 \end{document}

% Änderungen/Ergänzungen für einen Mitgliedschaftsantrag o.Ä. können gerne vorgeschlagen